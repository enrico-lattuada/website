%%%%%%%%%%%%%%%%%%%%%%%%%%%%%%%%%%%%%%%%%
% "ModernCV" CV and Cover Letter
% LaTeX Template
% Version 1.1 (9/12/12)
%
% This template has been downloaded from:
% http://www.LaTeXTemplates.com
%
% Original author:
% Xavier Danaux (xdanaux@gmail.com)
%
% License:
% CC BY-NC-SA 3.0 (http://creativecommons.org/licenses/by-nc-sa/3.0/)
%
% Important note:
% This template requires the moderncv.cls and .sty files to be in the same
% directory as this .tex file. These files provide the resume style and themes
% used for structuring the document.
%
%%%%%%%%%%%%%%%%%%%%%%%%%%%%%%%%%%%%%%%%%

%----------------------------------------------------------------------------------------
%	PACKAGES AND OTHER DOCUMENT CONFIGURATIONS
%----------------------------------------------------------------------------------------

\documentclass[11pt,a4paper,sans]{moderncv} % Font sizes: 10, 11, or 12; paper sizes: a4paper, letterpaper, a5paper, legalpaper, executivepaper or landscape; font families: sans or roman

\moderncvstyle{classic} % CV theme - options include: 'casual' (default), 'classic', 'oldstyle' and 'banking'
\moderncvcolor{blue} % CV color - options include: 'blue' (default), 'orange', 'green', 'red', 'purple', 'grey' and 'black'

%\usepackage{lipsum} % Used for inserting dummy 'Lorem ipsum' text into the template
\usepackage[T1]{fontenc}
\usepackage[utf8]{inputenc}
\usepackage[scale=0.75]{geometry} % Reduce document margins
%\setlength{\hintscolumnwidth}{3cm} % Uncomment to change the width of the dates column
\setlength{\makecvtitlenamewidth}{10cm} % For the 'classic' style, uncomment to adjust the width of the space allocated to your name
\usepackage[english]{babel}

%----------------------------------------------------------------------------------------
%	NAME AND CONTACT INFORMATION SECTION
%----------------------------------------------------------------------------------------

\firstname{Enrico} % Your first name
\familyname{Lattuada} % Your last name

% All information in this block is optional, comment out any lines you don't need
\title{Curriculum Vitae}
%\address{}{}
%\mobile{}
%\phone{+39 0297 84762}
%\fax{(000) 111 1113}
\email{enrico.lattuada@univie.ac.at}
\homepage{staff.org.edu/~jsmith}{staff.org.edu/$\sim$jsmith} % The first argument is the url for the clickable link, the second argument is the url displayed in the template - this allows special characters to be displayed such as the tilde in this example
\extrainfo{}%, italiano}
%\photo[70pt][0.4pt]{pictures/Fototessera} % The first bracket is the picture height, the second is the thickness of the frame around the picture (0pt for no frame)
%\quote{"A witty and playful quotation" - John Smith}

%----------------------------------------------------------------------------------------

\begin{document}
\flushbottom

\makecvtitle % Print the CV title

Date of birth: April 28th, 1990\\
Place of birth: Milano, Italy\\
Citizenship: Italian\\
My profile on scientific databases:
{\color{blue}\href{https://www.scopus.com/authid/detail.uri?authorId=57090180800}{Scopus}}, {\color{blue}\href{https://publons.com/researcher/1995890/enrico-lattuada/}{ResearcherID}}, {\color{blue}\href{https://scholar.google.it/citations?user=AWGMgbYAAAAJ&hl=en}{Google Scholar}}\\
%Number of published items: xx refereed papers + yy book chapters\\
H-index (Scopus, 01/02/2022): 4\\
Total citations (Scopus, 01/02/2022): 40\\

\textbf{Current address}\\
Faculty of Physics\\
University of Vienna\\
Boltzmanngasse, 5\\
1090, Vienna, Austria

%%%%%%%%%%%%%%%%%%%%%%%%%%%

\section{Education and career}

% Lise Meitner postdoc Vienna from 01/03/2022 to present
\cventry{Mar 2022 -- present}{Lise Meitner post-doctoral research fellow}{University of Vienna}{}{with Prof. Roberto Cerbino}{}
% Postdoc Roma from 03/12/2018 to 28/02/2022
\cventry{Dec 2018 -- Feb 2022}{Post-doctoral research fellow}{La Sapienza Universit\'a di Roma}{Roma (Italy)}{with Prof. Francesco Sciortino}{}
% Dottorato di ricerca from 02/11/2015 to 08/02/2019
% Thesis: Experimental study of the sedimentation of complex colloidal suspensions (Supervisor: Prof. Roberto Piazza)
\cventry{Nov 2015 -- Feb 2019}{Dottorato di Ricerca (Ph.D.) in Industrial Chemistry and Chemical Engineering}{Politecnico di Milano}{under the supervision of Prof. Roberto Piazza)}{\textit{cum laude}}{} 
% Junior research fellow intern from 22/06/2015 to 30/10/2015
\cventry{Jun 2015 -- Oct 2015}{Junior research fellow intern at Soft Matter Lab}{Politecnico di Milano}{}{under the supervision of Prof. Roberto Piazza}{}{}
% Laurea specialistica from 01/10/2012 to 29/04/2015
% Thesis: Study of the sedimentation of model colloidal suspensions using an analytical centrifuge (in Italian; Supervisor: Prof. Roberto Piazza)
\cventry{Oct 2012 -- Apr 2015}{Laurea specialistica (Master of Science) in Nuclear Engineering}{Politecnico di Milano}{}{\textit{110/110 cum laude}}{}
% Laurea triennale from 14/09/2009 to 26/09/2021
\cventry{Sep 2009 -- Sep 2012}{Laurea triennale (Bachelor of Science) in Energy Engineering}{Politecnico di Milano}{}{\textit{103/110}}{}
%\cventry{2004--2009}{Diploma di Liceo Scientifico}{Liceo Scientifico ''D. Bramante''}{Magenta (MI)}{indirizzo PNI}{}

%%%%%%%%%%%%%%%%%%%%%%%%%%%

\section{Schools}
\cventry{10-14/07/2017}{1st Summer School on Complex Fluid Flows in Microfluidics}{Universidade do Porto}{Porto -- Portugal}{}{}{}
\cventry{20-28/06/2022}{15th Bombannes Summer School on scattering applied to soft condensed matter}{}{Carcans-Maubuisson -- France}{}{}{}
\cventry{11-22/07/2022}{``Machine Learning for Materials Hard and Soft'' ESI-DCAFM-TACO-VDSP Summer School}{Erwin Schr\"odinger Institute}{Vienna -- Austria}{}{}{}

%%%%%%%%%%%%%%%%%%%%%%%%%%%

\section{Fellowships, grants \& awards}
\cventry{2022}{FWF (Austria Science Fund) Lise Meitner Post-doctoral Fellowship}{\textit{(177,980.00 €)}}{}{}{}{}
% Best poster award for Glancing at sedimenting invisible particles: a Ghost Particle Velocimetry Setup
\cventry{2018}{Best poster award}{}{Italian Soft Days 3rd edition, Padova (Italy)}{}{}{}
% Second best communication award for Non-equilibrium equation of state of a nanoparticle gel
\cventry{2017}{Second best communication award}{}{103rd National Congress of the Italian Physical Society, Trento (Italy)}{}{}{}
\cventry{2015}{MIUR (Italian Ministry of Education, University and Research) doctoral scholarship}{}{}{}{}{}

%%%%%%%%%%%%%%%%%%%%%%%%%%%

\section{Research interests}
\cvitem{}{$\circ$ Sedimentation}
\cvitem{}{$\circ$ Fluid dynamics}
\cvitem{}{$\circ$ Self-assembly and phase separation in complex colloidal suspensions}
\cvitem{}{$\circ$ Structure and dynamics of colloidal gels}
\cvitem{}{$\circ$ Optical techniques applied to soft matter}

%\cvitem{}{During my PhD at Politecnico di Milano, under the supervision of Prof. Roberto Piazza, I investigated the behaviour of gels and complex colloidal systems under the action of gravity (natural or forced, as when using a centrifuge). Using a colloidal system where the interactions can be quantitatively and finely tuned, I gave the first experimental confirmation that moderately concentrated suspensions of particles interacting \emph{via} strong attractive forces settle faster than a single isolated particle [P1]. The study is also relevant to the investigation of association effects in suspensions of weakly-interacting globular proteins, such as beta-lactoglobulin A.
%	I then focussed on the compressional behaviour of depletion gels subject to gravity (\textit{compressive rheology}, [P3]), on the development of a model colloidal system for the investigation of gel formation [P2] and on the settling velocity in bidisperse colloidal suspensions [P10].}
%\cvitem{}{As part of the study of the settling velocity of colloidal suspensions, I developed a hybrid scattering-imaging optical technique for the space-resolved velocimetry in suspensions of particles smaller than the optical resolution.
%	During my PhD, I also gained competences: (i) on the use of dynamic light scattering techniques, (ii) on the development of hybrid scattering-imaging techniques with coherent light (2D Photon Correlation Imaging, PCI) for the study of the dynamics and rheology of colloidal suspensions and gels and (iii) on the use of thermal lensing techniques, which I used in the last part of my project to study the self-assembly of block copolymers [P5].}
%\cvitem{}{Then, during my post-doc period at the University of Rome ``La Sapienza'', under the supervision of Prof. Francesco Sciortino (still ongoing), I took advantage of the acquired competences to experimentally study the self-assembly behaviour of DNA nanostars (NS), DNA nanostructures composed of multiple arms departing from a common centre which are designed to interact \emph{via} hybridisation of single-stranded DNA sticky sequences placed at the end of each arm.}
%\cvitem{}{I investigated systems of NS that self-assemble to mimic the formation of hyperbranched polymers -- highly branched aggregates -- using dynamic light scattering and simulations [P7]. The old famous Flory-Stockmayer theory predicts the formation of an infinite percolating cluster spanning the whole system when all bonds are formed. Our study allowed us to highlight the importance of intracluster bonds (links between NS that belong to the same aggregate, neglected by the theory) to prevent the formation of a percolating cluster.}
%\cvitem{}{Using 2D Photon Correlation Imaging, I experimentally investigated the spatially resolved dynamics of limited valence DNA NS gels. The results, published in Science Advances, demonstrate that these gels are structurally and dynamically uniform, hinting at the formation of a so-called equilibrium gel [P8].}
%\cvitem{}{During the post-doc period, I also had the opportunity to collaborate with a biology group to conduct a feasibility study on the use of DNA NS gels for biomedical applications [P6-9].}

\section{Publications}

%\cventry{[P10]}{\textbf{Settling velocity in bidisperse colloidal suspensions}}{}{\underline{E Lattuada}, S Buzzaccaro, A Parola, R Piazza}{\textit{(in preparation)}}{}
%[9]
\cventry{2022}{\textbf{Treatment of kidney clear cell carcinoma, lung adenocarcinoma and glioblastoma cell lines with hydrogels made of DNA nanostars}}{}{M Leo, \underline{E Lattuada}, D Caprara, L Salvatori, A Vecchione, F Sciortino, P Filetici, A Stoppacciaro}{\textit{Biomater. Sci.} (2022)}{}
%[8]
\cventry{2021}{\textbf{Spatially uniform dynamics in equilibrium colloidal gels}}{}{\underline{E Lattuada}, D Caprara, R Piazza, F Sciortino}{\textit{Sci. Adv.} \textbf{7} (2021), eabk2360}{}
%[7]
\cventry{2020}{\textbf{Hyperbranched DNA clusters}}{}{\underline{E Lattuada}, D Caprara, V Lamberti, F Sciortino}{\textit{Nanoscale} \textbf{12} (2020), 23003}{}
%[6]
\cventry{}{\textbf{DNA-GEL, novel nanomaterial for biomedical applications and delivery of bioactive molecules}}{}{\underline{E Lattuada}, M Leo, D Caprara, L Salvatori, A Stoppacciaro, F Sciortino, P Filetici}{\textit{Front. Pharmacol.} \textbf{11} (2020), 1345}{}
%[5]
\cventry{2019}{Thermophoresis in self-associating systems: Probing poloxamer micellization by opto-thermal excitation}{}{\underline{E Lattuada}, S Buzzaccaro, R Piazza}{\textit{Soft Matter} \textbf{15} (2019), 2140}{}
%[4]
\cventry{}{Compressive yield stress of depletion gels with variable interaction strength}{}{\underline{E Lattuada}}{\textit{Il Nuovo Cimento C} \textbf{42} (2019), 226}{}
%[3]
\cventry{2018}{Compressive yield stress of depletion gels from stationary centrifugation profiles}{}{\underline{E Lattuada}, S Buzzaccaro, R Piazza}{\textit{J. Phys.: Condens. Matter} \textbf{30} (2018), 044005}{}
%[2]
\cventry{2017}{Use of RAFT macro-surfmers for the synthesis of transparent aqueous colloids with tunable interactions}{}{U Capasso Palmiero, A Agostini, \underline{E Lattuada}, S Gatti, J Singh, CT Canova, S~Buzzaccaro, D Moscatelli}{\textit{Soft Matter} \textbf{13} (2017), 6439}{}
%[1]
\cventry{2016}{Colloidal Swarms Can Settle Faster than Isolated Particles: Enhanced Sedimentation near Phase Separation}{}{\underline{E Lattuada}, S Buzzaccaro, R Piazza}{\textit{Phys. Rev. Lett.} \textbf{116} (2016), 038301}{}


\section{Conference talks and posters}
%(invited talks are marked with a *)
\cventry{2022}{Spatially uniform dynamics in equilibrium colloidal gels}{presentation}{\underline{E Lattuada}, D Caprara, R Piazza, F Sciortino}{Polymer networks group international workshop}{Roma, Italy}
\cventry{2021}{Homogeneous dynamics in DNA equilibrium gels}{presentation}{\underline{E Lattuada}, D Caprara, R Piazza, F Sciortino}{35th ECIS Congress}{Athens, Greece}
\cventry{2020}{Hyperbranched DNA clusters}{presentation}{\underline{E Lattuada}, D Caprara, V Lamberti, F Sciortino}{Italian Soft Days, 4th edition}{Bari, Italy}
\cventry{2018}{Compressive yield stress of depletion gels from stationary centrifugation profiles}{presentation}{\underline{E Lattuada}, S Buzzaccaro, R Piazza}{Italian Soft Days,  3rd edition}{Padova, Italy}
\cventry{}{Glancing at sedimenting invisible particles: a Ghost Particle Velocimetry setup}{poster}{\underline{E Lattuada}, A Orlandini, S Buzzaccaro, R Piazza}{Italian Soft Days,  3rd edition}{Padova, Italy}
\cventry{2017}{Non-equilibrium equation of state of a nanoparticle gel}{communication}{\underline{E Lattuada}, S Buzzaccaro, R Piazza}{103o Congresso Nazionale della Società Italiana di Fisica}{Trento, Italy}
\cventry{}{Can colloidal swarms settle faster than isolated particles?}{presentation}{\underline{E Lattuada}, S Buzzaccaro, R Piazza}{10th Liquid Matter Conference}{Ljubljana, Slovenia}
\cventry{2016}{Can colloidal swarms settle faster than isolated particles?}{presentation}{\underline{E Lattuada}, S Buzzaccaro, R Piazza}{Italian Soft Days, 2nd edition}{Milano, Italy}
\cventry{}{Can colloidal swarms settle faster than isolated particles?}{presentation}{\underline{E Lattuada}, S Buzzaccaro, R Piazza}{3rd Workshop of the Complex Systems Group}{Milano, Italy}

%%%%%%%%%%%%%%%%%%%%%%%%%%%

\section{Students supervision and co-supervision}
\cventry{2021}{Tommaso Pietrangeli, Master candidate in Physics}{Department of Physics, Sapienza Università di Roma}{with Prof. Francesco Sciortino}{}{}
\cventry{2020}{Vincenzo Lamberti, Master candidate in Physics}{Department of Physics, Sapienza Università di Roma}{with Prof. Francesco Sciortino}{}{}
\cventry{2019}{Andrea Alessandrini, Master candidate in Nuclear Engineering}{Department of Chemistry, Materials Science, and Chemical Engineering, Politecnico di Milano}{with Prof. Roberto Piazza}{}{}
\cventry{2018}{Massimo Stefanoni, Master candidate in Nuclear Engineering}{Department of Chemistry, Materials Science, and Chemical Engineering, Politecnico di Milano}{with Dr. Stefano Buzzaccaro}{}{}
\cventry{2018}{Francesco Marafelli, Master candidate in Engineering Physics}{Department of Chemistry, Materials Science, and Chemical Engineering, Politecnico di Milano}{with Dr. Stefano Buzzaccaro}{}{}
\cventry{2018}{Andrea Orlandini, Master candidate in Chemical Engineering}{Department of Chemistry, Materials Science, and Chemical Engineering, Politecnico di Milano}{with Prof. Roberto Piazza}{}{}
\cventry{2018}{Andrea Francesco Mollame, Master candidate in Chemical Engineering}{Department of Chemistry, Materials Science, and Chemical Engineering, Politecnico di Milano}{with Dr. Stefano Buzzaccaro}{}{}
\cventry{2018}{Tommaso Botta, Batchelor of Science candidate in Engineering Physics}{Department of Chemistry, Materials Science, and Chemical Engineering, Politecnico di Milano}{with Dr. Stefano Buzzaccaro}{}{}
\cventry{2017}{Zeno Filiberti, Master candidate in Nuclear Engineering}{Department of Chemistry, Materials Science, and Chemical Engineering, Politecnico di Milano}{with Prof. Roberto Piazza}{}{}
\cventry{2017}{Alessandro Carbonaro, Master candidate in Engineering Physics}{Department of Chemistry, Materials Science, and Chemical Engineering, Politecnico di Milano}{with Prof. Roberto Piazza}{}{}
\cventry{2016}{Christopher Thomas Canova, ``Roberto Rocca'' Fellow visiting student from MIT (Dept. of Chemical Engineering)}{Department of Chemistry, Materials Science, and Chemical Engineering, Politecnico di Milano}{with Prof. Roberto Piazza}{}{}
\cventry{2016}{Roberto Pioli, Master candidate in Chemical Engineering}{Department of Chemistry, Materials Science, and Chemical Engineering, Politecnico di Milano}{with Dr. Stefano Buzzaccaro}{}{}
\cventry{2016}{Valentino Lepro, Master candidate in Biomedical Engineering}{Department of Chemistry, Materials Science, and Chemical Engineering, Politecnico di Milano}{with Prof. Roberto Piazza}{}{}
%\cventry{year--year}{Name}{institution}{with ...}{}{}

%%%%%%%%%%%%%%%%%%%%%%%%%%%

%\section{Teaching activity}

%%%%%%%%%%%%%%%%%%%%%%%%%%%

%\section{Organised events}

%%%%%%%%%%%%%%%%%%%%%%%%%%%

%\section{Institutional responsibilities}

%%%%%%%%%%%%%%%%%%%%%%%%%%%

%\section{Editorial activities}

%%%%%%%%%%%%%%%%%%%%%%%%%%%

\section{Languages}
% English:Council of Europe level B2 (FCE grade B)
\cvitem{}{Italian (native) and English (proficient).}

%%%%%%%%%%%%%%%%%%%%%%%%%%%

\section{Other activities}
\cvitem{}{Reviewer for \textit{Journal of Physics: Condensed Matter}, \textit{Soft Matter}, and \textit{Papers in Physics}.}
% Lead developer of ...

%%%%%%%%%%%%%%%%%%%%%%%%%%%

%\closesection{}
%\vspace{\fill}
%\footnotesize
%The undersigned is aware that, pursuant to art. 26 of Law 15/68, and Articles. 46 and 47 of Presidential Decree 445/2000, false statements, falsified acts and use of false acts are punishable under the Penal Code and special laws. Moreover, the undersigned authorizes the processing of personal data, in accordance with the provisions of Law 675/96 of 31 December 1996.


\end{document} 